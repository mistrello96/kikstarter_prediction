In questo capitolo discuteremo della qualità dei dati contenuti nei dataset utilizzati prima e dopo il processo di integrazione; inoltre, proporremo delle analisi descrittive effettuate sui dati integrati. 
\section{Descrizione dei dataset}
I 3 dataset utilizzati, che sono presenti nella \href{https://gitlab.com/Daniele-Papetti/kickstarterprediction}{repository online}, sono stati scaricati dal sito \href{https://www.kaggle.com/}{\emph{www.kaggle.com/}}.
In particolare, due di questi dataset (\textit{countries of the world.csv} e \textit{ks-projects-201612.csv}) sono quelli che verrano utilizzati effettivamente per estrarre delle features per il processo di apprendimento automatico, mentre il terzo (\textit{country\_code.csv}) viene utilizzato sia come dizionario per l'analisi di qualità dei dati, sia per la parte di integrazione.\\
Il dataset \textit{countries of the world.csv} contiene informazioni che descrivono il rispettivo territorio e sviluppo di 227 nazioni del mondo.
Tra queste si possono trovare informazioni utili come il PIL (GDP) \textit{pro capite}, lo sviluppo in percentuale dei tre settori ed il numero medio di telefoni per abitante; la chiave della relazione è il nome della nazione per esteso.
Invece, \todo[inline]{descrizione kickstarter e poi il perché del terzo dataset DP}
\section{Analisi di qualità dei dataset}
\todo[inline]{tutto DP}