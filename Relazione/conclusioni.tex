Concludendo, in questo lavoro si è partiti da 2 dataset, uno rappresentante le nazioni del mondo ed uno contenente le informazioni sui Kickstarter.
Per prima cosa, su questi dataset è stata effettuata una fase di analisi dei dataset sfruttando varie misure di qualità (ovvero, accuratezza sintattica, completezza, consistenza, leggibilità e misure temporali) che ha evidenziato un'ottima qualità dei dataset, ma al contempo ha evidenziato la necessità di fare attività di \textit{data fusion} tra il dataset delle nazioni ed il dizionario in modo da poi poter fondere senza problemi i due dataset di interesse.
Il processo di integrazione è stato poi concluso con successo, le misure di qualità del prodotto di questa fase lo confermano; il dataset prodotto è già un dataset pulito da un insieme di campi che non sarebbero risultati importanti per la fase di apprendimento automatico (\textit{i.e.}, il nome delle campagne).
Infine, sono state effettuate delle analisi descrittive sul dataset integrato in modo da comprendere come dei dati chiave per la fase successiva si distribuivano.\\
Per quanto concerne la parte di data technology, si possono considerare raggiunti a pieno tutti gli obiettivi.

Una volta che il dataset è stato prodotto, mediante il linguaggio \emph{R} sono stata prima eseguita un'ulteriore analisi esplorativa dei dati, successivamente si sono considerati i modelli per cui valutare le loro performance.
Oltre al modello baseline, utilizzato come modello di confronto, sono stati analizzati i comportamenti degli alberi decisionali, valutando le loro performance in vari casi (utilizzo di tutte le feature, non utilizzo della feature backers e alberi che utilizzano tutte le feature ma potati), il modello Na\"ive Bayes, le SVM e le reti neurali.
Di questi, per motivi di costi computazionali, è stato possibile effettuare un processo di 10-fold cross validation solo per gli alberi e il modello Bayesiano, invece, per le SVM e reti neurali sono state trainate con un trainset pari al 70 \% del dataset e testate con il complementare, questa scelta è stata possibile e ha permesso di tenere questi risultati in considerazione grazie all'elevata dimensione del dataset che garantisce che non si ricada in suddivisioni tra trainset e dataset “patologiche”.\\
L'analisi descrittiva ha prodotto dei risultati interessanti ma attesi, possiamo quindi considerarla soddisfatta; al contempo, anche se non è stato possibile effettuare una 10-fold cross validation per alcuni modelli la solidità del dataset ci ha permesso di confrontare le varie performance, quindi possiamo considerare anche il secondo obiettivo raggiunto.

Per quanto concerne gli sviluppi futuri, il lavoro ha fornito diversi punti di riflessione:\begin{itemize}
	\item ricercare dataset contenenti informazioni su come sono stati pubblicizzati i progetti Kickstarter, quindi integrarlo al dataset già esistente, effettuando le dovute analisi di qualità;
	\item utilizzare tale dataset per costruire un modello per effettuare regressione sulla feature backer, in modo da poter predirre tale numero utilizzando solo informazioni note all'avvio del progetto, in modo da poi poter usare tale risultato come feature al lavoro proposto in questo documento;
	\item analizzare il comportamento delle SVM con kernel diversi dal lineare e indagare più affondo le reti neurali.
\end{itemize}