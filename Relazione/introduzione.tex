\section{Dominio}
Il dominio di riferimento del progetto è quello delle startup, in particolare il sito di crowdfunding Kickstarter. Questa piattaforma permette a chiunque abbia un'idea da realizzare di proporla al pubblio, chiedendo il pagamento di diverse quote di denaro in cambio di ricompense crescenti e relative all'idea proposta.\\
Questo tipo di soluzioni hanno riscosso un discreto successo, producendo di fatto una grande quantità di informazioni circa l'interesse del pubblico per le varie tipologie di progetto.\\
L'idea scaturita dal nostro gruppo è stata quella di sfruttare i dati relativi ai progetti Kickstarter, resi disponibili pubblicamente sulla piattaforma \href{https://www.kaggle.com/}{\emph{Kaggle}}, al fine di comprendere quali aspetti di questi progetti potessero portare al loro successo o al loro fallimento (aspetti quali categoria, soglia di denaro richiesto, paese d'origine dell'idea \dots).

\section{Obiettivi}
Gli obiettivi di questo lavoro sono principalmente due, anche se a loro volta possono essere suddivisi in altri obiettivi: \begin{enumerate}
	\item realizzare un dataset integrato partendo da differenti dataset, per fare ciò sarà necessario: \begin{itemize}
		\item effettuare un'analisi di qualità sui singoli dataset;
		\item effettuare un processo di integrazione sfruttando anche i risultati dell'analisi precedente;
		\item analizzare la qualità del dataset prodotto per valutare se il processo di integrazione è stato buono oppure presenta ulteriori complicanze;
		\item eseguire delle analisi descrittive basilari per ottenere informazioni statistiche che potrebbe risultare rilevanti nella parte di apprendimento automatico.
	\end{itemize}
	\item partendo dal dataset integrato, effettuare un processo di apprendimento automatico per valutare le performance di vari modelli, risulta, quindi, necessario: \begin{itemize}
		\item effettuare un processo di analisi esplorativa del dataset;
		\item effettuare un processo di training e testing, sfruttando la \textit{K}-fold cross validation, per valutare i vari modelli e le relative performance.
	\end{itemize}
\end{enumerate}

\section{Selte di design}
\label{sec:design}
Al fine di creare un dataset integrato da poter sfruttare per eseguire il training del modello di Machine Learning, sono stati sfruttati tre diversi dataset provenienti da fonti diverse.\\
Per l'integrazione di questi dataset è stato necessario riconciliare alcune eterogeneità di schema. In particolare, all'interno del dataset contenente i dati relativi ai progetti Kickstarter, alcuni attributi mostravano eterogeneità di dominio, che hanno causato inizialmente diversi problemi in fase di analisi esplorativa del dataset.\\
È stato inoltre necessario eseguire operazioni di record linkage tra gli altri due dataset, in quanto gli attributi necessari ad eseguire l'operazione di join tra le due relazioni erano rappresentati mediante una sintassi simile ma non identica.